% beamer loads hyperref and url; let's pass our options to the package
\PassOptionsToPackage{unicode}{hyperref}
\PassOptionsToPackage{hyphens}{url}
\documentclass[
  ignorenonframetext,
  aspectratio=169
]{beamer}
\usepackage{nus-theme}

\usepackage[
  backend=biber,
  style=lncs,
]{biblatex}

\addbibresource{references.bib}

\newcommand{\citet}[1]{\citeauthor*{#1}~\cite{#1}}
\newcommand{\Citet}[1]{\Citeauthor*{#1}~\cite{#1}}

\hypersetup{
  pdftitle={Extending JumpProcesses.jl for fast point process simulation},
  pdfauthor={Guilherme Augusto Zagatti, Samuel A. Isaacson, Christopher Rackauckas, Vasily Ilin, See-Kiong Ng, Stéphane Bressan},
  pdflang={en},
  hidelinks,
  pdfcreator={Guilherme Zagatti}}

\title{Extending \texttt{JumpProcesses.jl}}
\subtitle{for fast point process simulation}
\author{%
  Guilherme Augusto Zagatti \inst{1} \and%
  Samuel A. Isaacson \inst{3} \and%
  Christopher Rackauckas \inst{4} \and%
  Vasily Ilin \inst{5} \and%
  See-Kiong Ng \inst{1,2} \and%
  Stéphane Bressan \inst{1,2}%
}
\institute{%
  \inst{1} Institute of Data Science, National University of Singapore \and%
  \inst{2} School of Computing, National University of Singapore \and%
  \inst{3} Department of Mathematics and Statistics, Boston University \and%
  \inst{4} Computer Science and AI Laboratory (CSAIL), Massachusetts Institute of Technology \and%
  \inst{5} Department of Mathematics, University of Washington
}
\date[JuliaCon 2023]{JuliaCon 2023}

\begin{document}
\begin{frame}
  \titlepage
\end{frame}

\hypertarget{introduction}{%
\section{Introduction}\label{introduction}}

\begin{frame}{Motivation}
% Don’t forget that your audience will likely not know much about jump processes in general, so a bit of an introduction to them and what they can be used for (and what the problem CoEvolve solves is) will likely be very helpful.

% applications in seismology, neuroscience, finance, healthcar, social media, biochemistry
% a simple model of chemical reaction?

\end{frame}

\hypertarget{point-processes}{%
\section{Point processes}\label{point-processes}}

\begin{frame}{Point processes}
% intro to point process; point vs jump processes; the compensator/rate
\end{frame}

\begin{frame}{Compensators}
\end{frame}

\begin{frame}{Contribution}
\end{frame}

\begin{frame}{Simulation}
\end{frame}

\hypertarget{jumpprocessesjl}{%
\section{JumpProcesses.jl}\label{jumpprocessesjl}}

\begin{frame}{DifferentialEquations.jl: Steppers and Callbacks}
% JumpProcesses; how does it connect to OrdinaryDiffEq
\end{frame}

\begin{frame}{JumpProcesses.jl Aggregator: A Custom DifferentialEquations.jl Callback}
\end{frame}

\begin{frame}{Simple simulation}
\end{frame}

\begin{frame}{Aggregator types}
% Algorithms <-> Aggregators
\end{frame}

\hypertarget{benchmarks}{%
\section{Benchmarks}\label{benchmarks}}

\begin{frame}{Benchmark Setup}
\end{frame}

\begin{frame}{Constant rate}
\end{frame}

\begin{frame}{The Hawkes process}
\end{frame}

\begin{frame}{Synapse}
\end{frame}

\hypertarget{conclusion}{%
\section{Conclusion}\label{conclusion}}

\begin{frame}{Conclusion}
\end{frame}

\begin{frame}
  \thispagestyle{empty@titlepage}
  \begingroup
    \setbeamercolor{structure}{fg=white}
    \begin{beamercolorbox}[sep=8pt,center]{title}%
      {\usebeamerfont{title}\usebeamercolor{title}Thank you!}%
    \end{beamercolorbox}
  \endgroup
\end{frame}

\begin{frame}[t,allowframebreaks]
  \frametitle{References}
  \printbibliography[heading=none]
\end{frame}

\end{document}
